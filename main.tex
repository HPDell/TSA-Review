\documentclass[UTF8,hyperref,a4paper,twoside]{ctexart}
    %------------------------------------------------------------------------------
% Ctex
%------------------------------------------------------------------------------
\setcounter{secnumdepth}{5}
\ctexset{
    section = {
        format = \Large\bfseries\raggedright
    },
    paragraph = {
        % name = {(,)},
        % number = \arabic{paragraph},
        numbering = false,
        % aftertitle = {\setcounter{subparagraph}{0}}
    },
    subparagraph = {
        % name = {(,)},
        % number = \alph{subparagraph},
        numbering = false,
        beforeskip = 0pt,
        % aftername = {},
        % indent = 0pt
    }
}

%------------------------------------------------------------------------------
% Unicode math
%------------------------------------------------------------------------------
\usepackage{amsmath}
\usepackage{amssymb}
\usepackage{unicode-math}
\setmainfont{XITS}
\setmathfont[bold-style = ISO]{XITS Math}


%------------------------------------------------------------------------------
% Minted
%------------------------------------------------------------------------------
\usepackage[outputdir=output]{minted}
\setminted{
    autogobble = true,
    linenos=true,
    frame=lines,
    framesep=2mm,
    fontsize=\scriptsize,
    breaklines=true
}
\setminted[py]{
    python3=true,
}
\setmintedinline{
    fontsize=\normalsize
}

%------------------------------------------------------------------------------
% Add new functions
%------------------------------------------------------------------------------
\usepackage{graphicx}
\usepackage{subfig}
% \usepackage{tikz}
\usepackage{multicol}

%------------------------------------------------------------------------------
% Define new macros
%------------------------------------------------------------------------------
\newcommand{\diff}{\symrm{d }}
\newcommand{\cpi}{\symrm{\pi}}
\newcommand{\trans}{^\symrm{T}}
\newcommand{\highlight}[1]{\textbf{\alert{#1}}}
\newcommand{\done}{\ooalign{$\square$ \cr \raisebox{3pt}{\scriptsize{$√$}}}}
\newcommand{\undone}{\ooalign{$\square$ \cr \raisebox{3pt}{}}}
\newcommand{\Cov}{\symrm{Cov}}
\newcommand{\Var}{\symrm{Var}}
\newcommand{\Corr}{\symrm{Corr}}
\newcommand{\ddfrac}[2]{\displaystyle{\frac{\displaystyle{#1}}{\displaystyle{#2}}}}

\newcommand{\AR}{\symrm{AR}}
\newcommand{\MA}{\symrm{MA}}
\newcommand{\ARMA}{\symrm{ARMA}}
\newcommand{\ARI}{\symrm{ARI}}
\newcommand{\IMA}{\symrm{IMA}}
\newcommand{\ARIMA}{\symrm{ARIMA}}

%------------------------------------------------------------------------------
% Format setting
%------------------------------------------------------------------------------
\usepackage[left=1.91cm, right=1.91cm, top=2.54cm, bottom=2.54cm]{geometry}
\hypersetup{colorlinks=true,bookmarksnumbered=true}
% \allowdisplaybreaks
\numberwithin{equation}{section}
\numberwithin{figure}{section}
\setlength{\abovecaptionskip}{0cm}
\setlength{\belowcaptionskip}{-0.5\baselineskip}
\setlength{\abovedisplayskip}{0\baselineskip}
\setlength{\belowdisplayskip}{0\baselineskip}
% 调整图标标题格式
\usepackage{caption}
\captionsetup{labelfont=bf,labelsep=quad}

    \title{\textbf{时间序列分析复习整理}}
    \author{\fangsong{胡奕公}}
    \date{\kaishu{2018年9月22日}}

    \begin{document}
        \maketitle

        \section{概念}

        \paragraph{时间序列} 是指将某种现象某一个统计指标在不同时间
        上的各个数值,按时间先后顺序排列而形成的序列。

        \paragraph{随机过程}必须用一族随机变量才能刻画该随机现象的全部统计规律性。 
        设 $ (ω, F, P) $ 是概率空间, $ T $ 是给定的参数集,若对每个 $ t ∈ T $ ,
        都 有一个随机变量  $ X(t,ω) $ 与之对应,则称随机变量族  $ \{ X(t,ω), t ∈ T \} $ 
        是 $ (ω, F, P) $ 上的随机过程,简记为随机过程 $ \{ X(t), t ∈ T \} $ 。
        $ T $ 称为参数集,一般表示时间。

        \paragraph{平稳性} 对其一切有限维分布函数随时间推移不发生改变的随机过程。
        自然地,决定过程特性的统计规律不随时间的变化而改变。

        \subparagraph{严平稳} 对于一切的时滞 $ k $ 和时点 $ t_1, t_2, ⋯, t_n $ ,
        都有 $ Y_{t_1}, Y_{t_2},  ⋯, Y_{t_n} $ 与
        $ Y_{t_{1−k}}, Y_{t_{2−k}}, ⋯, Y_{t_{n−k}}  $ 的联合分布相同。

        \subparagraph{弱平稳} 均值函数在所有时间上恒为常数;
        对所有的时间 $ t $ 和时滞  $ k $ , $$ γ_{t,t-k} = γ_{0,k} $$

        \paragraph{随机游动序列} 令 $ e_1, e_2, ⋯,  $ 为均值为 $ 0 $ ,方差 $ σ_e^2 $。
        观测时间序列 $ \{ Y_t, t = 1,2,⋯ \} $ 为 $$ Y_t = e_1 + e_2 + ⋯ + e_t $$ 
        或 $$ Y_t = Y_{t-1} + e_t $$ 

        \paragraph{滑动平均序列} 设 $ \{Y_t\} $  的构造如下
        $$ Y_t = \frac{e_t + e_{t-1}}{2} $$ 

        \paragraph{随机趋势} 随机游动序列似乎具有普通的上升趋势;然而,
        \begin{itemize}
            \item 随机游动在任何时间点都有零均值
            \item 随机游动的方差随时间增加而增加
            \item 对同一过程进行再次模拟可能会展现完全不同的“趋势”。
        \end{itemize}

        \paragraph{确定性趋势} 对同一过程进行再次模拟不会展现完全不同的“随机性”。

        \paragraph{二阶矩过程} 设 $ {x(t)} $ 为一(实值或复值)随机过程,
        且对每个 $  t ∈T, {x(t)}  $ 的 均值和方差都存在,则称 $ {x(t)} $ 为二阶矩过程。

        \paragraph{一般线性过程} 成现在和过去白噪 声变量的加权线性组合
        $$ Y_t = e_t + ψ_1 e_{t-1} + ψ_2 e_{t-2} + ⋯
               = ∑_{j=-∞}^∞ ψ_j e_{t-j} $$
        
        \paragraph{滑动平均过程}  $ q $ 阶滑动平均过程
        $$ Y_t = e_t - θ_1 e_{t-1} - θ_2 e_{t-2} - ⋯ - θ_q e_{t-q} $$

        \paragraph{自回归过程}  $ p $ 阶自回归过程
        $$ Y_t = ϕ_1 Y_{t-1} + ϕ_2 Y_{t-2} + ⋯ + ϕ_p Y_{t-p} + e_t $$ 

        \subparagraph{特征多项式} 
        $$ ϕ (x) = 1 - ϕ_1 x - ϕ_2 x^2 - ⋯ - ϕ_p x^p $$ 

        \subparagraph{特征方程} 当且仅当 AR 特征方程的所有根都在单位圆外时,特征方程存在平稳解。
        $$ 1 - ϕ_1 x - ϕ_2 x^2 - ⋯ - ϕ_p x^p = 0 $$

        \paragraph{自回归滑动平均模型}  $ \mathrm{ARMA}(p,q) $
        $$ Y_t = ϕ_1 Y_{t-1} + ϕ_2 Y_{t-2} + ⋯ + ϕ_p Y_{t-p} + e_t
               - θ_1 e_{t-1} - θ_2 e_{t-2} - ⋯ - θ_q e_{t-q} $$ 
        
        \paragraph{可逆性} 滑动平均过程可以重新表示为自回归模型。

        \paragraph{对数变换} 如果序列的散度变大似乎与序列值的增加有关联,
        即序列的值越大,围绕该值的波动也越大。

        \paragraph{幂变换} 只有对正值数据才可作幂变换,
        对于有部分非负数据的序列,需整体加个正数使所有数据为正。
        $$ g(x) = \left\{ \begin{array}{ll}
            \frac{x^λ - 1}{λ} & λ ≠ 0 \\
            \log (x) & λ = 0
        \end{array} \right. $$ 

        \paragraph{样本自相关函数}
        $$ r_k = \ddfrac{
            ∑_{t=k+1}^n (Y_t - \bar{Y})(Y_{t-k} - \bar{Y})
        }{
            ∑ _{t=1}^{n} (Y_t - \bar{Y})
        } $$

        \paragraph{样本偏自相关函数}
        其为 $ Y_t $ 与 $ Y_{t−k} $ 之间的消除介入变量 $ Y_{t−1}, Y_{t−2}, ⋯, Y_{t−k+1}$
        的影响后的相关系数函数。偏自相关函数即是预测误差之间的相关系数。
        $$ ϕ_{kk} = \ddfrac{
            ρ_k - ∑ _{j-1}^{k-1} ϕ_{k-1,j} ρ_{k-j}
        }{
            1 - ∑ _{j-1}^{k-1} ϕ_{k-1,j} ρ_{j}
        } $$ 
        其中, $ ϕ_{k,j} = ϕ_{k-1,j} - ϕ_{kk}ϕ_{k-1,k-j}, j = 1, 2, k-1 $ 。

        \paragraph{扩展的样本自相关系数} 设 
        $$ W_{t,k,j} = Y_t - \tilde{ϕ}_1 Y_{t-1} - ⋯ - \tilde{ϕ}_k Y_{t-k}  $$
        为 AR 系数定义的自回归残差,其系数估值 $ \tilde{ϕ} $ 是在设 AR 阶数为 $ k $ 、
        MA 阶数为 $ j $ 时迭代估计所得, $ W_{t,k,j} $ 的样本自相关系数称为扩展的样本自相关系数。

        \paragraph{ADF 检验} 用最小二乘回归所得估计系数的 $ t $ 统计量作为检验统计量,
        在有单位根的零假设下,该检验统计量服从某种非标准的大样本分布。

        \paragraph{AIC 准则} 估计模型与真实模型的平均 Kullback-Leibler 偏离的估计量,定义为
        $$ \mathrm{AIC} = -2 \log(\mathrm{MLE}) + 2k $$
        AIC 准则要求选择使 AIC 最小化的模型;
        $ 2k $ 相当于一个“惩罚函数”,可避免过度参数化,确保选择简洁模型;
        当模型包含常数项时,取 $ k = p + q + 1 $ ,否则取 $ k = p + q $。

        \paragraph{修正的AIC准则} 在 AIC 中增加另一个非随机的惩罚项,使之近似消除 偏差。
        定义为: $$ \mathrm{AIC_C} = \mathrm{AIC} + \frac{2(k+1)(k+2)}{n-k-2} $$

        \paragraph{BIC 准则} BIC 准则确定 ARMA 模型阶数的方法是
        选择一个能使 Schwarz 贝叶斯信息准则最小的模型。其定义为
        $$ \mathrm{BIC} = -2 \log(\mathrm{MLE}) + k \log{n} $$

        \paragraph{条件平方和函数} 最小二乘估计要优化的函数
        $$ S_c(ϕ, μ) = ∑ _{i=2}^{n} [(Y_t - μ) - ϕ (Y_{t-1} - μ)]^2 $$
        
        \paragraph{无条件平方和函数} 似然函数的一部分
        $$ S_(ϕ, μ) = ∑ _{i=2}^{n} [(Y_t - μ) - ϕ (Y_{t-1} - μ)]^2
                         + (1 - ϕ^2)(Y_1 - μ) $$
        
        \paragraph{Box Pierce 统计量} 当估计为正确的 ARMA 模型时,对于大样本 $ n $ ,
        $ Q $ 近似服从自由度为 $ K - p - q $ 的 $ \chi^2 $ 分布。
        $$ Q = n(\hat{r}_1^2 + \hat{r}_2^2 + ⋯ + \hat{r}_K^2) $$

        \paragraph{Ljung Box 统计量} 
        $$ Q_⋆ = n(n+2) \left( 
            \frac{\hat{r}_1^2}{n-1} + \frac{\hat{r}_2^2}{n-2} + ⋯ + \frac{\hat{r}_K^2}{n-K}
        \right) $$ 

        \paragraph{过度拟合} 识别并拟合出合适的模型之后,再拟合一个“接近”模型,
        该模型以 原始模型为特例包容原始模型。如果
        \begin{enumerate}
            \item 额外的参数估计不显著地为 0; 
            \item 共同的参数估计与原始的估计相比没有显著的改变。
        \end{enumerate}
        则认为时正确的模型。

        \paragraph{最小方差预测} 预测均方误差最小。

        \paragraph{确定性趋势预测} 简言之,预测既是将确定性时间趋势外推至未来时刻点上。
        $$ \hat{Y}_t (l) = μ_{t+l} $$
        
        \paragraph{ARMA 预测} 
        \subparagraph{AR(1) 模型} $$ Y_t(l) = ϕ^l (Y_t - μ) + μ $$ 
        预测误差
        $$ e_t(1) = Y_{t+1} - \hat{Y}_t(1) = e_{t+1} $$ 
        $$ e_t(l) = e_{t+l} + ϕ e_{t+l-1} + ⋯ + ϕ^{l-1} e_{t+1} $$ 
        预测精度
        $$ \Var [e_t(l)] = σ_e^2 \left[ \frac{1 - ϕ^{2l}}{1 - ϕ^2} \right] → \frac{σ_e^2}{1 - ϕ^2} $$ 
        
        \subparagraph{MA(1) 模型} 一步预测及其误差为
        $$ \hat{Y}_t(1) = μ - θ e_t $$
        $$ e_t(1) = Y_{t+1} - \hat{Y}_t(1) = e_{t+1} $$
        $ l $ 步预测为 $$ \hat{Y}_t(l) = μ $$ 
        误差为  $$ e_t(l) = e_{t+l} + ψ e_{t+l-1} + ⋯ + ψ_{l-1} e_{t+1} $$ 
        精度为 $$ \Var [e_t(l)] = σ_e^2 (1+ψ_1^2 + ψ_2^2 + ⋯ + ϕ_{l-1}^2) $$ 

        \subparagraph{带漂移的随即游动} 预测
        $$ \hat{Y}_t(l) = Y_t + θ_0 l $$
        误差 $$ e_t(l) = e_{t+1} + e_{t+2} + ⋯ + e_{t+l} $$ 
        精度 $$ \Var [e_t(l)] = l σ_e^2 $$ 

        \subparagraph{ARMA 模型}

        \paragraph{非平稳模型}

        \paragraph{预测极限} 给定置信度水平 $ 1- α $ ,可用标准正太百分位数 $ z_{1 - α /2} $ 证明
        $$ P \left( -z_{1-α/2} < \ddfrac{Y_{t+l} - \hat{Y}_t(l)}{\sqrt{\Var(e_t(l))} < z_{1-α/2}} \right) = 1 - α $$ 
        因此未来观测值 $ Y_{t+l} $ 落在观测极限 $ \hat{Y_t}(l) ± z_{1-α/2}\sqrt{\Var(e_t(l))} $ 
        中的置信度为 $ (1-α) 100\% $
        
        \paragraph{乘法季节模型} 如 $ P = q = 1, p = Q = 0, s = 12 $ 的模型为
        $$ Y_t = Φ Y_{t - 12} + e_t - θ e_{t - 1} $$

        \paragraph{干预分析} 干预分析提供了对干预影响进行评估的框架,
        干预是通过改变均值函数或趋势而影响过程,
        干预可能改变时间序列的自协方差函数。

        \paragraph{单一干预} 单一干预模型 $$ Y_t = m_t + N_t $$ 其中,
        $ m_t $ 表示均值变化, $ N_t $ 表示均值函数变化。假设序列在 $ T $ 时刻发生干预,
        当 $ t<T $ 时, $ m_t = 0 $ 。 $ \{Y_t, t < T\} $ 是预干预数据。

        \paragraph{干预类型} 
        \subparagraph{阶梯函数}  $ S_t^{(T)} = 1_{t ⩾ T} $ 
        \subparagraph{脉冲函数}  $ P_t^{(T)} = S_t^{(T)} - S_{t-1}^{(T)} = 1_{t = T} $ 
        
        \paragraph{干预的影响}
        \subparagraph{立即起效}  $ m_t = ω S_t^{(T)} $ ,换为脉冲函数为短期干预。
        \subparagraph{延迟起效}  $ m_t = ω S_{t-d}^{(T)} $ 
        \subparagraph{逐渐影响}  $ m_t = δ m_{t-1} + ω S_{t-d}^{(T)} $ 即
        $$ m_t = 1_{t ⩾ T} ⋅ ω \frac{1 - δ^{t-T}}{1 - δ} $$ 
        当 $ 1 - δ^{t-T} = 0.5 $ 即 $ t = T + \frac{\log 0.5}{\log δ} $  时,
        $ m_t $ 达到极限变化值的一半, $ \frac{\log 0.5}{\log δ} $ 称为半衰期。
        $ δ $ 越小,半衰期越小,系统越快达到极限变化值。换为脉冲函数即为逐渐消失的影响。

        \paragraph{异常值} 不规则观测值,源于测量误差与复制误差其中之一或两者皆有,亦或过程突发短期性变化。 
        \subparagraph{可加异常值}  $ Y'_t = Y_t + ω_A P_t^{(T)} $ 
        \subparagraph{新息异常值} $ e'_t = e_t + ω_I P_t^{(T)} $ 

        \paragraph{IO检验} 用于检验 $ T $ 时刻IO检验统计量为
        $$ λ_{1,T} = \frac{a_t}{σ} $$ 
        其中 $$ a_t = Y'_t - π_1 Y'_{t-1} - π_2 Y'_{t-2} - ⋯ $$
        若序列只在 $ T $ 时刻有IO,则残差为 $ a_T = ω_I + e_T $ ,其它情况下
        $ a_t = e_t $ ,因此 $ ω_l $ 可由 $ \hat{ω}_I = a_t $ 来估计,方差为 $ σ^2 $ 。

        \paragraph{AO检验} 假设只有 $ T $ 时刻存在 AO,则可证 $$ a_t = -ω_A π_{t-T} + e_t $$ 
        $ ω_A $ 的最小二乘估计量为
        $$ \hat{ω}_{T,A} = -ρ^2 ∑ _{t=1}^{n} π_{t-T} α_t $$
        其中 $ ρ_2 = (1 + ϕ_1^2 + π_2^2 + ⋯ + π_{n-T}^2)^{-1} $ ,
        估计量方差为 $ ρ^2 σ^2 $
        统计量为 $$ λ_{2,T} = \frac{\hat{ω}_{T,A}}{ρσ} $$

        \paragraph{互协方差函数} $ Y=\{ Y_t \} $ , $ X=\{ X_t \} $ ,互协方差函数为
        $$ γ_{t,s}(X,Y) = \Cov (X_s, Y_t) $$
        联合平稳过程中 $ X $ 与 $ Y $ 在滞后 $ k $ 的互相关函数为
        $$ ρ_k(X,Y) = \Corr (X_t, Y_{t-k}) = \Corr (X_{t+k},Y_t) $$
        互相关函数一般都不是偶函数。

        \paragraph{样本互协方差函数} 
        $$ r_k(X,Y) = \ddfrac{
            ∑ (X_t - \bar{X})(Y_{t-k} - \bar{Y})
        }{
            \sqrt{∑ (X_t - \bar{X})^2} \sqrt{∑ (Y_t - \bar{Y})^2}
        } $$

        \paragraph{回归模型} $$ Y_t = β_0 + β_1 X_{t-d} + e_t $$
        修正的回归模型 $$ Y_t = β_0 + β_1 X_{t-d} + Z_t $$
        其中 $ Z_t $ 也是一个 $ \mathrm{ARIMA}(p,d,q) $ 模型。

        \paragraph{预白化} 一般地,若 $ Xt $ 满足某可逆 $ \ARIMA(p,d,q) $ 模型,则有 $ \AR(∞) $ 形式
        $$ \tilde{X}_t = (1 - π_{1}B - π_{2}B^2 + ⋯ ) X_t = π(B) X_t  $$
        其中 $ \tilde{X}_t $ 是白噪声,上述过程称为预白化。

        \paragraph{傅里叶级数} 以 $ T $ 为周期的周期函数 $$ f(t) = f(t+T) $$
        如果在周期 $ \left( -\frac{T}{2}, \frac{T}{2} \right) $ 内满足 Dirichlet 条件,
        就可以展开称正交函数线性组合的无穷级数——傅里叶级数。
        \begin{enumerate}
            \item 连续或只有有限个第一类间断点(左右极限具有有限值,间断点取值是左右极限的平均值)
            \item 具有有限个极值点。
        \end{enumerate}

        \paragraph{三角级数} 对于满足 Dirichlet 条件的任意周期信号,在 $ \left( -\frac{T}{2}, \frac{T}{2} \right) $
        上可展开成三角级数
        $$ y(t) = a_0 + ∑ _{k=1}^{∞} \left[ a_k \cos(2πf_k t) + b_k \sin (2π f_k t) \right] $$
        直流分量 $$ a_0 = \frac{1}{T} ∫_{\frac{T}{2}}^{\frac{T}{2}} y(t) \diff t $$
        余弦分量 $$ a_k = \frac{2}{T} ∫_{\frac{T}{2}}^{\frac{T}{2}} y(t) \cos(2πf_k t) \diff t $$
        正弦分量 $$ a_k = \frac{2}{T} ∫_{\frac{T}{2}}^{\frac{T}{2}} y(t) \sin(2πf_k t) \diff t $$
        如果时间离散化,变求积分为求和。

        \paragraph{傅里叶频率} 时间离散化, $ T = nτ $ ,则 $ f = \frac{1}{n τ } $ , $ f_k = \frac{k}{nτ} $
        形如 $ \frac{1}{n}, \frac{2}{n}, ⋯, \frac{K}{n}  $ 称为傅里叶频率。

        \paragraph{周期图} 频率 $ f = \frac{j}{n} (j = 1, 2, ⋯, n) $ 上周期图 $ I $ 的定义为
        $$ I(\frac{j}{n}) = \frac{n}{2} \left( \hat{A}_j^2 + \hat{B}_j^2 \right) $$
        当样本量为偶数时,在极端频率 $ f = \frac{k}{n} = \frac{1}{2} $ 上,
        $$ I(\frac{j}{n}) = n \left( \hat{A}_k^2 \right) $$

        \paragraph{谱表示} 任何零均值平稳过程都可用连续频带上的无穷多个正余弦的线性组合表示成
        $$ Y_t = ∫_{0}^{\frac{1}{2}} \cos(2πft) \diff a(f) + ∫_{0}^{\frac{1}{2}} \sin(2πft) \diff b(f) $$
        的形式。其中
        $$ a(f) = ∑_{\left\{ j | f_j ⩽ f \right\}} A_j $$
        $$ b(f) = ∑_{\left\{ j | f_j ⩽ f \right\}} B_j $$
        是以 $ 0 ⩽ f ⩽ \frac{1}{2} $ 上的频率为自变量的零均值随机过程,其各自的增量相互正交(不相关);
        并且 $ a(f) $ 和 $ b(f) $ 的增量也不相关。

        \paragraph{谱分布函数}
        $$ \Var \left( ∫_{f_1}^{f_2} \diff a(f) \right) = \Var \left( ∫_{f_1}^{f_2} \diff b(f) \right) 
           = F(f_2) - F(f_1) $$
        非降函数 $ F(f) $ 称为谱分布函数。对于 $ 0 ⩽ f ⩽ \frac{1}{2} $ 平稳过程有线谱
        $$ F(f) = ∑_{\left\{f|f_j ⩽ f\right\}} σ^2_j $$

        \paragraph{谱密度} 若协方差函数绝对可加,则对于 $ -\frac{1}{2} < f ⩽ \frac{1}{2} $
        理论谱密度为
        $$ S(f) = γ_0 + 2 ∑_{k=1}^{∞} γ_k \cos(2πfk) $$
        逆运算
        $$ γ_k = ∫_{-\frac{1}{2}}^{\frac{1}{2}} S(f) \cos(2πfk) $$

        \paragraph{时不变线性滤波器} 时间序列 $ \{X_t\} $ 经过绝对可加常熟序列
        $ ⋯ , c_{-1}, c_0, c_1, c_2, c_3, ⋯  $ 滤波后,生成新的序列 $ \{Y_t\} $
        $$ Y_t = ∑ _{j=-∞}^{∞} $$ 若对 $ k<0 $ 有 $ c_k = 0 $ ,则称该滤波器有因果性。

        \paragraph{ARMA 过程的谱密度}
        \subparagraph{白噪声} 谱密度为常数 $$ S(f) = \sigma_e^2 $$
        \subparagraph{$ \MA(1) $过程} $$ S(f) = \left[ 1 + \theta^2 - 2\theta \cos(2 \pi f) \right] \sigma_e^2 $$
        \subparagraph{$ \MA(2) $过程}
        $$ S(f) = \left[1+\theta_1^2+\theta_2^2-2\theta_1(1-\theta_2)\cos(2\pi f)-2\theta_2\cos(4\pi f)\right] \sigma_e^2 $$
        \subparagraph{$ \AR(1) $过程} $$ S(f) = \frac{\sigma_{e}^{2}}{1 + \phi^2 -2\phi \cos(2\pi f)} $$
        \subparagraph{$ \AR(2) $过程}
        $$ S(f) = \frac{\sigma_{e}^{2}}{
            \left[1+\phi_1^2+\phi_2^2-2\phi_1(1-\phi_2)\cos(2\pi f)-2\phi_2\cos(4\pi f)\right]
        } $$
        \subparagraph{$ \ARMA(1,1) $过程}
        $$ S(f) = \frac{1 + \theta^2 - 2 \theta \cos(2\pi f)}{1 + \phi^2 - 2 \phi \cos(2\pi f)} \sigma_e^2 $$
        \subparagraph{季节$ \AR(1) $过程}
        $$ S(f) = \frac{\sigma_{e}^{2}}{
            \left[1 + \phi^2 -2\phi \cos(2\pi f)\right]
            \left[1 + \Phi^2 -2\Phi \cos(2\pi 12f)\right]
        } $$
        \subparagraph{季节$ \MA(1) $过程}
        $$ S(f) = 
        \left[1 + \theta^2 -2\theta \cos(2\pi f)\right] 
        \left[1 + \Theta^2 -2\Theta \cos(2\pi 12f)\right] $$

        \paragraph{样本谱密度}
        \begin{itemize}
            \item 样本谱密度不能估计理论谱密度
            \item 样本谱密度是理论谱密度的无偏估计
            \item 样本谱密度方差丝毫不依赖于样本量 $ n $ 。样本谱密度不是理 论谱密度的一致估计。
            \item 样本谱密度是相当一般的渐近无偏估计,但也 是不一致的,作为可用的估计存在极大的可变性。
        \end{itemize}

        \paragraph{谱密度估计方法}
        \subparagraph{参数估计} 假设自回归模型(可能为高阶),给出时间序列的“最 好”拟合,估计其谱密度。
        这是基于被拟合AR模型的理论谱密度。
        \subparagraph{非参数估计} (基本)无需事先修定“真实”谱密度的形状,亦 称样本谱密度的平滑。 

        \paragraph{Daniell 窗} 
        \subparagraph{简单平均} 以某傅立叶频率 $ f $ 为中心向两边各扩展 $ m $ ,对 $ 2m + 1 $ 个样本谱值取平均,
        即得平滑样本谱密度:
        $$ \bar{S}(f) = \frac{1}{2m+1} \sum_{j=-m}^{m} \hat{S}(f+\frac{j}{n}) $$
        在接近 $ 0 $ 和 $ \frac{1}{2} $ 的频率处,按周期图处理。
        \subparagraph{加权平均} 选取一个加权函数 $ W_m(f) $ 平滑谱密度估计为
        $$ \bar{S}(f) = \frac{1}{2m+1} \sum_{j=-m}^{m} W_m(k) \hat{S}(f+\frac{j}{n}) $$

        \paragraph{修正的 Daniell 窗} 过降低极值点权重,能够有效克服Daniell窗端点的剧烈变化。
        或者多次应用谱窗对周期图进行平滑。在数 学意义上,该做法相当于应用了所有谱窗的卷积。

        \paragraph{带宽} $$ \symrm{BW} = \frac{1}{n} \sqrt{\sum_{k=-m}^{m} k^2 W_m(k)} $$

        \paragraph{泄露} 主峰两侧有明显的“旁瓣”,这会导致非傅立叶频率上的功率泄露到附近傅立叶频率上的功率中。

        \paragraph{锥削} 通过降低序列两端数据的幅度,使其数值慢慢趋于数据的均值零。 
        最常用的形式基于余弦钟,更常用的锥削由分裂余弦钟给出,即仅将余弦锥削应用于时间序列的极点而得到。

        

    \end{document}